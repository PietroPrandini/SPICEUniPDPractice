\documentclass[10pt,a4paper]{book}
\usepackage{amsmath}
\usepackage{amsfonts}
\usepackage{amssymb}
\usepackage[english]{babel}
\usepackage{float}
\usepackage[left=2cm,right=2cm,top=2cm,bottom=2cm]{geometry}
\usepackage{graphicx}
\usepackage{hyperref} % Used for external links
\usepackage[utf8]{inputenc}
\usepackage{listings} % Used for source code listing
\usepackage{mathtools}

\setcounter{tocdepth}{3}

% Source code listing's parameters
\lstset{
  frame=single,
  keepspaces=true,
%  title=\lstname
}

\title{Third SPICE Exercise\\{\small{Fundamentals Of Electronics - a.a. 2018-2019 -
University of Padua (Italy)}}}
\author{Pietro Prandini (mat. 1097752)}

\begin{document}
\maketitle

\vspace*{\fill}
% License
\begin{center}
\tiny{This work is licensed under the Creative Commons Attribution-ShareAlike 4.0 International License. To view a copy of this license, visit \href{http://creativecommons.org/licenses/by-sa/4.0/}{http://creativecommons.org/licenses/by-sa/4.0/} or send a letter to Creative Commons, PO Box 1866, Mountain View, CA 94042, USA.}
\end{center}

\tableofcontents

\chapter{Differential amplifier with MOS current source}\label{diffampchapter}

\begin{figure}[h]
  \centering
  \includegraphics[width=12cm]{schematics/DifferentialAmplifier.jpg}
  \caption{Differential amplifier with MOS current source}
  \label{DifferentialAmplifier}
\end{figure}

Initial data:\\
\begin{align}
V_t = 0.5V\\
{K'}_n = {\mu}_n C_{ox} = 200 \frac{\mu A}{V^2}\\
\lambda = 0\\
\left(\frac{W}{L}\right)_1 = \left(\frac{W}{L}\right)_2 = 20\\
\left(\frac{W}{L}\right)_3 = \left(\frac{W}{L}\right)_4 = 5\\
R_{D_1} = R_{D_2} = 20k\Omega\\
R_{D_4} = \frac{30}{1000}\cdot 1097752 \Omega = 32.93k\Omega \simeq 33k\Omega\\
V_{DD} = 3V\\
V_{SS} = -3V
\end{align}

\section{Static conditions - Analytic solution}

\begin{figure}[h]
  \centering
  \includegraphics[width=12cm]{schematics/DifferentialAmplifierStaticConditions.jpg}
  \caption{Differential amplifier with MOS current source - Static conditions}
  \label{DifferentialAmplifierStaticConditions}
\end{figure}

On static conditions it's considered the input signals $V_{i_1}$ and $V_{i_2}$ turned off.\\
The equivalent circuit is on figure \ref{DifferentialAmplifierStaticConditions}.\par

\subsection{MOSFET $M_4$}
\subsubsection{Saturation mode checks}\label{M4SatCheck}
The transistor $M_4$ has a short circuit between its drain and its gate, so the transistor works in saturation mode and the voltage between the drain and the gate are the same of the voltage between the gate and the source:\par
\begin{align}
V_{{D_4}SS} > V_{{G_4}SS} - V_t \xRightarrow{V_{{G_4}SS} = V_{{D_4}SS}}
V_{{D_4}SS} &> V_{{D_4}SS} - V_t\\
0 &> - V_t \quad \text{(Always true: }V_t > 0\text{ )}
\end{align}
It's requested to pay attention to the another check to confirm the work on saturation mode:
\begin{gather}
V_{{G_4}SS} > V_t \xRightarrow{V_{{G4}SS} = V_{{D_4}SS}} V_{{D_4}SS} > V_t \label{CheckSat2eq}
\end{gather}


\subsubsection{$V_{D_4SS}\quad (=V_{G_4SS})$}
Supposing that the transistor $M_4$ works on the saturation mode (see the section \ref{M4SatCheck} for details), the current $I_{D_4}$ could be calculated as:\\
\begin{align}
I_{D_4} &= \frac{1}{2}{K'}_n \left(\frac{W}{L}\right)_4 (V_{D_4SS} - V_t)^2 \label{ID4Sat}
\end{align}
Other expression of the current $I_{D_4}$ could be calculated using the LKT:\\
\begin{align}
V_{DD}-R_{D_4}I_{D_4}-V_{D4SS}-V_{SS} = 0 \implies
I_{D_4} = \frac{V_{DD} - V_{D_4SS} - V_{SS}}{R_{D_4}}\label{ID4K}
\end{align}
Using the equations \ref{ID4Sat} and \ref{ID4K} it's possible calculating $V_{D_4SS}$:\\
\begin{align}
\frac{1}{2}{K'}_n \left(\frac{W}{L}\right)_4 (V_{D_4SS} - V_t)^2 = \frac{V_{DD} - V_{D_4SS} - V_{SS}}{R_{D_4}}
\end{align}
\begin{align}
\frac{1}{2}\cdot 200 \frac{\mu A}{V^2} \cdot 5 \frac{\mu m}{\mu m} (V_{D_4SS} - 0.5V)^2 = \frac{3V - V_{D_4SS} - (-3V)}{33k\Omega}\\
500 \frac{\mu A}{V^2} (V_{D_4SS} -0.5V)^2 = \frac{6}{33}mA-\frac{1}{33k\Omega}V_{D_4SS}\\
500 \frac{\mu A}{V^2} (V_{D_4SS}^2- V_{D_4SS}\cdot V +0.25V^2) = \frac{6}{33}mA-\frac{1}{33k\Omega}V_{D_4SS}\\
500 \frac{\mu A}{V^2} \cdot V_{D_4SS}^2 + \left(-500 \frac{\mu A}{V^2}V + \frac{1}{33k\Omega}\right) V_{D_4SS} + 500 \frac{\mu A}{V^2} \cdot 0.25V^2 -\frac{6}{33}mA = 0\\
0.5 \frac{mA}{V^2} \cdot V_{D_4SS}^2 + \left(-0.5 \frac{mA}{V^2}V + \frac{1}{33k\Omega}\right) V_{D_4SS} + 0.5 \frac{mA}{V^2} \cdot 0.25V^2 -\frac{6}{33}mA = 0\\
\left(0.5 \frac{mA}{V^2}\right) V_{D_4SS}^2 + \left( -\frac{31}{66}\frac{mA}{V}\right) V_{D_4SS} + \left(-\frac{5}{88}mA\right) = 0
\end{align}
\begin{align}
V_{{D_{4}SS}_{1,2}} = \frac{-\left( -\frac{31}{66}\frac{mA}{V}\right)\pm \sqrt{\left(-\frac{31}{66}\frac{mA}{V}\right)^2-4\cdot\left(0.5 \frac{mA}{V^2}\right)\cdot\left(-\frac{5}{88}mA\right)}}{2\cdot0.5 \frac{mA}{V^2}} = 
\left\{
\begin{array}{l}
1.04784V \label{VD4SS}\\
-0.10845V \quad \text{Not possible: }<\text{ of } V_t\\
\end{array}
\right. 
\end{align}

Now it's possible to check the last equation that can confirm the work on saturation mode of the MOSFET $M_4$ (equation \ref{CheckSat2eq}):
\begin{align}
1.04784V > 0.5V \quad M_4 \text{ works on saturation mode.} \label{M4SatConfirm}
\end{align}

\subsubsection{$I_{D_4}$}
Using the equation \ref{ID4Sat} and the result of the equation \ref{VD4SS}:
\begin{align}
I_{D_4} = \frac{1}{2}\cdot 200 \mu A/V^2\cdot 5 \frac{\mu m}{\mu m} \cdot \left(1.04784V -0.5V \right)^2 = 150.06433 \mu A \label{ID4SatResult}
\end{align}

\subsection{MOSFET $M_3$}
\subsubsection{$V_{G_3SS}$}\label{VG3SS}
Observing the circuit represented on the figure \ref{DifferentialAmplifierStaticConditions} it's clear that the voltage $V_{G_3SS}$  is equal to the voltage $V_{D_4SS}$ calculated in the equation \ref{VD4SS}.
\begin{align}
V_{G_3SS} = V_{D_4SS}
\end{align}

\subsubsection{$I_{S_A}$}
As agree with the consideration of the section \ref{VG3SS} and supposing the work of the MOSFET $M_3$ on the saturation mode, it's possible calculating the drain current of the MOSFET $M_3$:\\

\begin{align}
I_{SA} = \frac{1}{2}{K'}_n \left(\frac{W}{L}\right)_3 (V_{D_4SS} - V_t)^2
\end{align}
\begin{align}
I_{SA} = \frac{1}{2}\cdot 200 \mu A/V^2\cdot 5 \frac{\mu m}{\mu m} \cdot \left(1.04784V -0.5V \right)^2 = 150.06433 \mu A \label{ISASat}
\end{align}

\subsubsection{Saturation mode checks}
For obtaining the confirm of the work of the MOSFET $M_3$ on saturation mode the next two equations have to be satisfied:\\
\begin{align}
V_{{D_3}SS} > V_{{G_3}SS} - V_t \xRightarrow{V_{{D_3}SS} = V_{S_ASS}, V_{{G_3}SS} = V_{{D_4}SS}} V_{{S_A}SS} > V_{{D_4}SS} - V_t \label{M3SatCheckVSASS}
\end{align}
\begin{align}
V_{{G_3}SS} > V_t \xRightarrow{V_{{G_3}SS} = V_{{D_4}SS}} V_{{D_4}SS} > V_t \label{M3SatCheckVD4SS}
\end{align}
The equation \ref{M3SatCheckVSASS} hasn't to be checked, $V_{S_ASS}$ isn't calculated yet.\\
The equation \ref{M3SatCheckVD4SS} is satisfied (see the equation \ref{M4SatConfirm}).\par

\subsection{MOSFET $M_1$ and MOSFET $M_2$}
The MOSFET $M_1$ and the MOSFET $M_2$ have the same dimension and the same constructive parameters (see initial data at the start of this chapter \ref{diffampchapter}).\\
They also have the same voltage applied to every their pins (see figure \ref{DifferentialAmplifierStaticConditions}).\\
So, additionally supposing the work on the saturation mode of $M_1$ and $M_2$, it's possible to confirm the next equations:\\
\begin{align}
I_{D_1} &= I_{D_2}\label{ISA/2}\\
V_{G_1S_A} &= V_{G_2S_A}\\
V_{D_1S_A} &= V_{D_2S_A}
\end{align}

\subsubsection{$I_{D_1} (= I_{D_2})$}
\begin{align}
\text{LKC node } S_A \text{: } I_{D_1} + I_{D_2} - I_{S_A} = 0 \xRightarrow{eq. \ref{ISA/2}} 2I_{D_1} - I_{S_A} = 0 \implies I_{D_1} = \frac{I_{S_A}}{2}
\end{align}
\begin{align}
I_{D_1} = \frac{150.06433 \mu A}{2} = 75.03217 \mu A
\end{align}

\subsubsection{$V_{G_1S_A} ( = V_{G_2S_A})$}
Supposing the work of the MOSFET $M_1$ (equally $M_2$) on the saturation mode, the drain current could be calculated as:\\
\begin{align}
I_{D_1} = \frac{1}{2} K'_n \left(\frac{W}{L}\right)_1 (V_{G_1S_A} - V_t)^2 \label{ID1eq}
\end{align}
It's possible using the equation \ref{ID1eq} to calculate $V_{G_1S_A}$:\\
\begin{align}
I_{D_1} &= \frac{1}{2} K'_n \left(\frac{W}{L}\right)_1 (V_{G_1S_A} - V_t)^2\\
\sqrt{I_{D_1}} &= \sqrt{\frac{1}{2} K'_n \left(\frac{W}{L}\right)_1} (V_{G_1S_A} - V_t)\\
\sqrt{\frac{I_{D_1}}{\frac{1}{2} K'_n \left(\frac{W}{L}\right)_1}} &= V_{G_1S_A} - V_t\\
V_{G_1S_A} &= \sqrt{\frac{2I_{D_1}}{K'_n \left(\frac{W}{L}\right)_1}} +V_t\\
V_{G_1S_A} &= \sqrt{\frac{2 \cdot 75.03217 \mu A}{200 \mu A \cdot 20 \frac{\mu A}{\mu A}}} +0.5V\\
V_{G_1S_A} &= 0.69369V \label{VG1SAvalue}
\end{align}

\subparagraph{$V_{S_A}$}
Now it's possible calculating the voltage on the node $S_A$:
\begin{align}
V_{G_1S_A} = V_{G_1} - V_{S_A} \implies V_{S_A} = V_{G_1} - V_{G_1S_A}
\xRightarrow{V_{G_1} = 0, eq. \ref{VG1SAvalue}} V_{S_A} = -0.69369V
\end{align}

\subsubsection{Saturation mode checks}
In order to check the mode of the $M_1$ (and $M_2$) the equations to respect are \ref{M12checkSatA} and \ref{M12checkSatB}.\\
\begin{align}
V_{D_1S_A} &> V_{G_1S_A} - Vt \label{M12checkSatA}\\
V_{D_1} - V_{S_A} &> V_{G_1S_A} - Vt\\
I_{D_1}R_{D_1} - V_{S_A} &> V_{G_1S_A} - Vt\\
75.03217 \mu A \cdot 20k\Omega - (-0.69369V)&> 0.69369V - 0.5V\\
2.19433V &> 0.19369V \quad \text{True.}
\end{align}
\begin{align}
V_{G_1S_A} &> V_t \label{M12checkSatB}\\
0.69369V &> 0.5V \quad \text{True.}
\end{align}


Checking the mode of the $M3$'s work (see equation \ref{M3SatCheckVSASS}):
\begin{align}
V_{{S_A}SS} &> V_{{D_4}SS} - V_t\\
V_{S_A} - V_{SS} &> V_{{D_4}SS} - V_t\\
-0.69369V - (-3V) &> 1.04784V - 0.5V\\
2.30631V &> 0.54784V \quad M_3\text{ works on the saturation mode.}
\end{align}

\subsection{MOSFET $V_{DS_Q}$, $V_{GS_Q}$, $I_{D_Q}$ - Resuming}
\begin{center}
\begin{tabular}{|c|c|c|c|}
\hline
MOSFET & $V_{DS_Q}$ & $V_{GS_Q}$ & $I_{D_Q}$ \\
\hline
$M1$ & $V_{D_1S_A} = 2.19433V$ & $V_{G_1S_A} = 0.69369V$ & $I_{D_1} = 75.03217\mu A$ \\
\hline
$M2$ & $V_{D_2S_A} = 2.19433V$ & $V_{G_2S_A} = 0.69369V$ & $I_{D_2} = 75.03217\mu A$ \\
\hline
$M3$ & $V_{S_ASS} = 2.30631V$ & $V_{D_4SS} = 1.04784V$ & $I_{S_A} = 150.06433 \mu A$ \\
\hline
$M4$ & $V_{D_4SS} = 1.04784V$ & $V_{D_4SS} = 1.04784V$ & $I_{D_4} = 150.06433 \mu A$ \\
\hline
\end{tabular}
\end{center}

\subsection{$g_m$}
\begin{align}
g_{m_1} = g_{m_2} &= K'_n \left(\frac{W}{L}\right)_1 (V_{G_1S_A} - V_t)\\
&= 200 \frac{\mu A}{V^2} \cdot 20 \frac{\mu A}{\mu A} \cdot (0.69369V-0.5V)\\
&= 774.76 \mu A/V
\end{align}
\begin{align}
g_{m_3} = g_{m_4} &= K'_n \left(\frac{W}{L}\right)_3 (V_{D_4SS} - V_t)\\
&= 200 \frac{\mu A}{V^2} \cdot 5 \frac{\mu A}{\mu A} \cdot (1.04784V-0.5V)\\
&= 547.80 \mu A/V
\end{align}

\subsection{MOSFET $M_3$ with $\lambda = 0.02$}
From now the MOSFET $M_3$ is considered with $\lambda = 0.02$.\par
On this way there are some changes of the voltages and the currents of the circuit but they could be considered negligible, so it's considered true the past result from now too.\par
With $\lambda = 0.02$, the $r_0$ has a finite value:\\
\begin{align}
r_0 &= \frac{1}{\lambda I_{S_A}}
= \frac{1}{0.02 \cdot 150.06433 \mu A}
\simeq 333.2k\Omega
\end{align}

\subsection{Differential gain}


\section{SPICE analysis}
\subsection{Operating Point on static conditions}
\lstinputlisting{netlist/DifferentialAmplifierStaticConditions.cir}
\lstinputlisting{netlist/DifferentialAmplifierStaticConditions.op}

\subsection{Operating Point - common mode signal}
\lstinputlisting{netlist/DifferentialAmplifierVCM.cir}
\lstinputlisting{netlist/DifferentialAmplifierVCM.op}

\subsection{Operating Point - differential signals}
\lstinputlisting{netlist/DifferentialAmplifierVID.cir}
\lstinputlisting{netlist/DifferentialAmplifierVID.op}



\end{document}
